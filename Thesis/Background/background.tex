\chapter{BACKGROUND}
\label{chap:background}

\nomenclature[z23]{\pd}{Proportional Derivative}
\nomenclature[z24]{\lc}{Limit Cycle}
\nomenclature[z25]{\cpg}{Central Pattern Generator}
\nomenclature[z26]{\eph}{Equilibrium Point Hypothesis}
\nomenclature[z27]{\umh}{Uncontrolled Manifold Hypothesis}
What differs \cms methods are different ideas of motor control.
Current \cms research adopted the control hierarchy for many artificial system,
where there is a clear separation of planning and execution.
Body are treated as mechanical apparatus, which execute the motion command planned by the neural system.

Motor Invariant Theory(\moit) is based on the integrative theory of motor control\citep{dickinson2000animals}:
there is no a clear separation between planning and execution,  motor control can only be understood as a whole.

In this chapter, limitations of current \cms research are discussed first, which are the motivation of this research.
\moit is developed because such limitations can not be overcome without breaking the framework.
Foundational biological research are discussed later,  which serve as justification for \moit.



\section{A survey of \cms}

Many methods are developed in \cms research and it is impossible to include all the research work and discuss everyone in details.

In this short discussion,\cms methods are categorized by the control model: memory or computational.
Memory model inspired the data-driven techniques;
procedure methods are computation based.
Pros and cons are discussed category by category.

\subsection{Data Driven}
Data-driven methods are based on ready motion data which are generated by Key-frame or Motion Capture(Mocap). 
In practice, motion data are segmented into short time clips. 
An animation is synthesized by selecting motion clips and connecting them together\citep{Parent2002,kovar2003flexible}.

Like other example based methods, data driven methods can generate good results if similar motion clips are available, but difficult to generate  adaptation or novel motion, whether for a different character or scenario. 
This is the  ``re-targeting'' problem.

Besides the limitation in adaptation,  data management is another problem in practice. 
The Annotation Database \citep{Arikan2003} and the Motion Graph \citep{kovar2008motion} are proposed. 
Currently, catalogue and search of motion data are not trivial and remain open\citep{keogh2004indexing,muller2005efficient}.

\subsection{Procedural Method}
For physics based \cms, different procedural approaches have been proposed.
\begin{itemize}
\HiItem{Tracking Controllers}




Some early research applied classical \pd controller \citep{Raibert1991} for dynamic motion synthesis.
Later research \citep{Hodgins1995} applied the same method for different tasks like running, bicycling, vaulting and balancing. 
For high dimensional characters, \pd controller tracking the predefine motion curves\citep{Yin2007}.

\pd controller is shown in Equation~\ref{eq:pdcontrol}.
\begin{equation}
\label{eq:pdcontrol}
u=K(q -q_d)+d\qd
\end{equation}
where $u$ is the control effort, $K$ is the stiffness, $q_d$ is the desired or reference position, $d$ is the damping efficient.
\pd control based method can run in real-time and generating adaptive responses to small perturbation.
Large perturbation response or deviation from the reference trajectory are difficult to achieve with \pd controller.


Most \pd based controllers use motion capture data as references.
\citet{Laszlo1996} introduced Limit Cycle (\lc) as tracking reference for periodic locomotion animation. 
In current researches\citep{Coros2009,coros2010generalized,Laszlo1996} track fixed limit cycle,
such methods share many characteristics with \pd,it can run in really time, but motion result has limited adaptation ability and looks stereotype.







\HiItem{Optimization}
Because of the redundant \dof s, motion planning is non-deterministic.
For the problem, optimization has been introduced.
The idea is among all the possible motions, the ``best'' one is chosen.

Many metric has been proposed, 
for dynamic methods, a reasonable merit is the energy cost~$E$. 
\begin{equation}
 \textbf{E}=\int_{t_0}^{t_1}f_{a}(t)^2dt
\end{equation}
where $f_{a}$ is the active force generated by actuators like motors or muscles. 
This is introduced to \cms research as the influential Spacetime Constraints\citep{Witkin1988}. 
It is based on the hypothesis that the natural looking trajectory costs minimum energy. 
It is related to the idea of Darwin's Theory of Evolution and the principle of Natural Selection. 
Optimization based  methods produced very believable motions for variable tasks. 
\citet{Jain2009} provides an example of locomotion.  
\citet{BalanceControl} find a method for balance maintaining movement. 
\citet{Liu2009} proposed a method for object manipulating animation. 
\end{itemize}


\subsubsection*{Drawbacks of Optimization}
Optimization is the current mainstream method for physics based animation.
It generated the best motion results in current research.
But this method has several drawbacks.

\begin{itemize}
\HiItem{Numerical Stability and Modelling Difficulties: }
Optimization methods promise the energy efficiency of the resulting motion, but no grantee about convergence speed and stability.
Optimal solution is difficult to find numerically, and are sensitive to the accuracy of the model and the proximity of the initial guess.
\citet{Liu2005} points out those space–time constraint methods only suit high energy motions, like jumping and running.
For low energy tasks (such as walking) the results do not look natural.

\HiItem{Computational Complexity: }
Optimization with space–time constraints is a variational problem by nature. 
For a complex characters, it might takes  prohibitively long time, limiting the application domain of problems to those which are computationally feasible. 
In addition, little is known about how to reuse a computation result for motion adaptation.
\end{itemize}


\subsection{Fix Up}




\subsection{Biological Constraints}
The problems of \cms has also been spotted earlier by biological motor control.
The theory from tradition artificial systems such as \pd or optimization, are highly unlikely the principles for biological motor control, for they  violates the biological constraints.
Although the mechanism behind information processing remains obscure, some characteristics of biological information processing are well agreed, which make  \cms methods above questionable\citep{Glynn2003}. 
  
\begin{itemize}
\HiItem{Sensing and Control Limitations:}
Motor control is not only a mechanical problem, but a complex process involves chemical, electrical changes.
Many crucial mechanical parameters and variables such as mass, inertia, force, are inaccessible to the neural system and can only be approximated. 
For important control variables (such as torque), the neural system controls indirectly through a complex process.
Also body and environmental measurements are noisy and time varying, making methods that are sensitive to errors unsuitable for biological motor control.

\HiItem{Neural Computation: }
The neural system is powerful, but is inferior in speed and accuracy when compared with a digital computer. 
It can only generate signals at hundreds of Hz, signal transmission speeds are slow, and there is a long delay between firing a neural signal and generating force in the muscles.
it may cost about half a second from seeing an object to force generation in arm, . 
This makes it impossible for the neural system to carry out the complex computation necessary for real–time optimization.


Following the idea of optimization control, the dynamics of fluid environment and deformable body are more difficult to optimize. 
But most primitive life forms live in the sea and have limited intelligence. 
\HiItem{Memory Capacity:}
Some people argue that motion control is not based on computation, but based on memory.
This idea may helps to drop the question of computation speed, but it faces the memory capacity problem. 
Motion varies greatly, if we store the motion in our brain, the problems is the memory capacity.
\end{itemize}

\section{Motion Primitives}
Many animals include human exhibit complex motion behaviours at very young age.
Many complex motion behaviour like breathing, heat beating and child bearing are inborn ability without the need for learning.
These evidence suggests that motor ability may organized in blocks\citep{bizzi1995modular,bizzi2002book}.
Strong evidences come the experiment of stimulating of a single spinal motor afferent triggers a complete sweeping motion\citep{bizzi1995modular}.
The number of motion primitives is limited.
Complex motions are combinations of motion primitives, just like we connect alphabets into sentences.
Such building blocks are called \emph{motion primitives}.
Motion primitives are also give insight into the motion perception.
\citet{gallese1996action} have found action and perception trigger similar reactions in a group of neurons.





\subsection{Dynamic Motion Primitives}
It is impossible in reproduce the whole body and neural system in computer to produce character motions.
For dynamic \cms, the key question is how motion primitives simplify the dynamics of motor control.

An alternative idea that animals don’t move the way they want, but rather the way they can. 
Motion style is not changed much by the neural system evolution, after all whale swim more like fish than other mammals.
The motion style is close related to the body structure and environment.
These finding lead us shift our focus away from the neural system and understand motor control through an integrative view which incorporate neural system, body structure and environment\citep{dickinson2000animals}.
The body and the environment play the most important role in motor control, as they form the basic pattern of motion \citep{nishikawa2007neuromechanics}.


Follows the questions of motion primitive models.
Motion primitive should not be a trajectory tracking system.
For even under the same conditions, the motions still vary. 
Some \dof s are not controlled and freely influenced by the environment. 
\emph{Uncontrolled Manifold Hypothesis(UMH)}\citep{latash2008neurophysiological} propose in motor control, trajectory is not concern, only the final results is.


\emph{Equilibrium Point Hypothesis(EPH)}\citep{Feldman1986} is a specification of UMH for . 
This idea comes from properties of differential equations. 
For a dynamic system
\[
\dot{\state}=F(\state)
\]
the equilibrium points $\state_{e}$ satisfy the condition $F(\state_{e})=0$.
EPH suggests that what the neural systems controls is not trajectory, but the equilibrium points.



\emph{Impedance Control} \citep{hogan1985ica} refines the idea of EPH by providing an explanation for effects of the extra \dof s. 
At an equilibrium point $\state_{e}$,
\[
F(\state_{e})=0 
\]
Impedance Control proposed that the extra \dof s provide a way to control the stability and admittance of the equilibrium point $_{e}$. 
The mathematical presentation is
\begin{equation}
F(\state_{e}+E_r)=KE_r
\end{equation}
where $E_r$ is the offset error vector, $K$ is stiffness matrix or impedance,will determines the stability.
Neural system will tune the direction of $K$ according to the motion purpose, such as avoiding obstacles and risks. 
Experiment \citep{Franklin2007} shows that the matrix $K$ has anisotropic properties.







\subsection{Neural Control Model}
Human motor control involves little mental work.
The current idea of biology research is that motor control is a low level intelligent activity and can be controlled  without brain input. 
Two models are developed for ``tweaking'' motion primitives.
\begin{itemize}
\item
In vertebrate animals,  Central Pattern Generator (\cpg) serves many functions in locomotion, respiration and swallowing and other rhythm behaviour.
\citet{Cohen1988a} argues that locomotion is the result of the interaction between neural and mechanical oscillators via a process called \textbf{entrainment}.
Neural systems modify the motion by adjust frequency and amplitude of neural rhythmic signal.



\item
Some research find motion will change in a uniform manner\citep{Viviani1992},\citep{flash2007affine} propose modelling motion adaptation through \emph{affine transformation}.
Both works implies close relationship between motor control and the vision system.
\end{itemize}









\subsection{ Evidences from bionic Robotic Research}
Biological research idea greatly inspired the engineering experiment.
Some researches begin to focus on utilizing the natural dynamic follow the biological motor control principle.
And some significant result has been reported
\begin{itemize}
\HiItem{Limit Cycle in Walking}
A very important discovery is the bipedal walking can happen without any control\citep{McGeer1990}.  
And based on this idea, new mechanical system is designed that can walk on plane with simple control strategy\citep{Collins2005}.

\HiItem{\cpg and entrainment}
The \cpg based entrainment is applied for robotic research\citep{Williamson1999a}, the finding results show the \cpg will boost the system stability and can maintain motion in unpredictable situation.

\HiItem{Control Symmetry}
Symmetry is also well exploit in Mechanical research as conservative laws.
The idea of Symmetry is also used in control robotics\citep{spong2005controlled}.
\end{itemize}

\section{Placement and Contrasts}
This biological research ideas are the foundation upon which the motor invariant theory is built.
The mathematical concept of topological conjugacy is introduced to generalize and unify different ideas.

Motion Primitives comes from the \emph{structual stable} component in natural dynamics.
\eph and Impedance Control are generalized as control of attractor and attraction.

\cpg and Transformation are different in principle, there is no research attempts to unify the two ideas in one motor control framework.
Motor Invariant Theory include both methods for a good biological reason: \cpg comes from the research of functions of spinal cord, which models the low level control; while the transformation idea comes from search of the cortex, which is a model for high level control.
 
Within Motor Invariant Theory,  \cpg and transformation works as complementation.
\cpg boost the stability of motion primitives though entrainment, it works as a low level qualitative control mechanism.
Transformation adapt motion for task specific purpose, it is more precise and serve as quantitative control.

Motor Invariant Theory also extend current biological research by include motion primitives transition.


 

 



