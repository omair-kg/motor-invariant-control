
% Thesis Abstract -----------------------------------------------------


%\begin{abstractslong}    %uncommenting this line, gives a different abstract heading
\begin{abstracts}        %this creates the heading for the abstract page

Generating natural-looking motions for virtual characters is a challenging research topic, made even harder when adapting synthesized motions to interact with the environment. 
Current methods are tedious to use, computational expensive and fail to capture natural looking features.
These difficulties seem to suggest that artificial control techniques are inferior to the natural counterparts.

Recent advances in biology research point to an alternative idea:
The motion repertoire consists of a limited number of elements: the motion primitives. 
Complex motions are synthesized by connecting motion primitives together, just like connecting alphabets into sentences.
These elements serve as templates, which are tweaked by the neural system to satisfy  environmental constraints or motion purpose.

%we propose principle of motor control is not feedback based, they should by model as topology conjugacy,mechanical system to form an analogous dynamic system that meets constraints and purpose.

Based on the idea of motion primitive,   this thesis propose a new method to generate natural-looking motion.
The insight is that the formation motion primitives has a deep dynamic reason: synthesized motions should be stable. 
When the neural system construct motions from the primitives, many  valuable properties such as stability and efficiency are preserved, which are called \emph{motor invariants}.


New mathematical tools are introduced to character animation research.
Invariant Theory ,especially mathematical concepts of equivalence and symmetry, lay the theriotic foundation.
Invariant qualitative properties of motion primitives are identified by their  differential topology.
Adaptation is mathematically modelled as topological conjugacy: a transformation action which maintains the topology and results in an analogous system.

The \emph{Neural Ocillator} and \emph{Symmetry Preserving Transformations} are proposed to motion synthesis for their computational efficiency.
Even without reference motion data, this approach produces natural looking motion in real-time, and might  provide insights into biological motion  control and perception .

\end{abstracts}
%\end{abstractlongs}


% ----------------------------------------------------------------------


%%% Local Variables: 
%%% mode: latex
%%% TeX-master: "../thesis"
%%% End: 

