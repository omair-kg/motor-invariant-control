
% Thesis Abstract -----------------------------------------------------


%\begin{abstractslong}    %uncommenting this line, gives a different abstract heading
\begin{abstracts}        %this creates the heading for the abstract page

Generating natural-looking motions for virtual characters is a challenging research topic, made even harder when adapting synthesized motion to interac with the environment. 
Current methods are tedious to use, computational expensive and fail to capture natural looking features.
These difficulties suggested artificial control techniques are inferior to the natural counterparts.

Recent advances in biology research points to an alterantive theory:
The motion repertoire consists of a limited number of elements called motion primitives. 
Complex motions are synthesized by connecting motion primitives together just like connecting alphabets into sentences.


Motion primitives serve as templates from which motion is constructed.
Neural system tweaks these templates to generate motion which satisfies  environmental constraints or motion purpose.
In this process, some properties of motion primitives are maintained, which are called motor invariants.

%we propose principle of motor control is not feedback based, they should by model as topology conjugacy,mechanical system to form an analogous dynamic system that meets constraints and purpose.

In this thesis I propose an efficient method of generating natural-looking motion based on these principles. 
I introduce new mathematical tools for identifying and tweaking motion primitives to adapt to environmental conditions.
Motor invariants are identified by using the mathematical concepts of equivalence and symmetry.
Motion Primitives are identified by the qualitative properties, captured by the differential topology.
I propose that motion adaption is modelled as topological conjugacy: a transformation which maintains the topology and result in an analogous system.
To this end, the neural Ocillator and Symmetry Preserving transformations are proposed as computational efficient methods for this purpose.

Even without reference motion data, this approach produces natural looking motion in real-time and may provide insights into biological motion perception and control.

\end{abstracts}
%\end{abstractlongs}


% ----------------------------------------------------------------------


%%% Local Variables: 
%%% mode: latex
%%% TeX-master: "../thesis"
%%% End: 

