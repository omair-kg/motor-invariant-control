\chapter{A Review Of Previous of CMS Research}
\label{chap:previouswork}

\section {The Key Questions of CMS Research}
Motion Synthesis is a research topic that aggregates many different ideas.
At first, we discuss the common questions before comparing the variety of ideas.
From our point of view, two key questions are shared by all the CMS research:
The model and the algorithm.

For any CMS research, 
Motion needs a mathematical description.
Basically we have three models for motion.
\begin{description} 
\item [Parametric Curve]
Motion can be described as a parametric curve with time as the parameter and position as value. 
For example, and arm movement can be described by positions of all the joints during the time of motion.
\item [State Machine]
Basically, complex motion and be divided into several primary motions.
Motion can be described by discrete model such as state machine, like reach the apple and then pick it up.
\item [Objective]
Motion can be simple described by the objective, like picking up an apple.
\end{description}

Another question shared by CMS research is the algorithm for searching motion solution.
Current idea separates this task into two levels:
The high level problem is motion planning, its responsibility is to finds a solution.
Low level problem is motion maintenance or stability control.
The task is maintaining motion under perturbations.
Efficiency is always the concern in algorithm research.

It is impossible to discuss algorithms and model separately.
The model should be treated as the foundation of a CMS research.
It determines quality and difficulty.
Usually, model without detailed information like state machine may ease the searching problem for motion solution, 
but limited information will make the low level control nondeterministic;
while a detailed motion  model will make high level planning more difficult, 
more information  may help low level control, but may also be over specific that no solution can be found.
There is an elaborate balance.

For most current research, 
Motion is described as trajectories or parametric curves. 
Motor control is transformed into two tasks: find a trajectory and follow it. 
Based on different input information, trajectory based methods can be classified in two groups,
\textbf{Data Driven} and \textbf{Procedural}.

\section{Data Driven Methods} 
Data driven methods are based on ready motion data which are generated by Key-Frame or Motion Capture(Mocap). 
In practice, motion data are segmented into short time clips. 
An animation is generated by selecting motion clips and connecting them together\citep{Parent2002}.

Like other example based methods in Computer Graphics, data driven methods can generate good results if similar motion clips can be found. 
However it is difficult to generate novel motions, 
it is also difficult to reuse the motion data, whether modifying the motion data for a different character or a different scenario. 
This is the ``motion re-targeting'' problem.

In practice, the management of large motion data is another big challenge. 
Although some solutions like the Annotation Database \citep{Arikan2003} and the Motion Graph \citep{kovar2008motion} were proposed, cataloguing and searching of motion data are not trivial tasks and remain an open problems.

\section{Procedural Methods}
For procedural methods, pre-recorded motion data are not needed. 
The motions are formulated as functions of various parameters, as shown in equation \ref{eq:motion_func}.
\begin{equation}
\label{eq:motion_func}
F:P \mapsto M
\end{equation}
where $M$ is the valid motion space of characters, $P$ is the parameter space.

Different models may have different choices of parameters $P$ and mapping function $F$. 
Currently, $P$ and $F$ are mainly based on rules of physical science, and such procedural methods are often called ``Simulation Methods''. 

For kinematic methods, $P$ is mapped to the kinematic properties like position and shape.
$F$ is based on kinematic laws.  
Position errors such as foot sliding on the ground can be easily eliminated by kinematic methods. 

For dynamic methods, $P$ is expanded to include dynamic properties like mass, energy and force, $F$ is based on mechanic theory of Newton or Lagrange.
Since natural object movement is governed by mechanics, this research direction has the potential to generate more realistic motions. 
For dynmaic methods, a dynamic model is needed for the character body. 
A common model is a linked rigid body mechanism.
After rigid body dynamic simulation methods\citep{Baraff1994,Mirtich1996,Stewart2000} were proposed, comes the prosperity of dynamic CMS research. 

Rigid body simulation is just one of the many problems CMS faces. 
The bodies of animals are actuated by muscles under the control of neural system. 
Control system design is the most difficult problem.

Some early research applied classical control methods like PD controller \citep{Raibert1991} for locomotion synthesis.
Later research \citep{Hodgins1995} applied the same method for different tasks like running, bicycling, vaulting and balancing. 
Such methods are based on simplified models which relieved the controllers from the problem of redundant DOFs, 
but important motion details were also neglected.
\section{Optimization Methods}
\label{sec:optimization}
Because of the redundant DOFs in the body structure, in most cases, there exist many motion solutions for one task.
Optimization methods have been applied to solve the nondeterministic problem. 
Among all the solutions in possible motion space, the ``best'' one is chosen as the proper solution:
\begin{equation}
\label{equ:max_select}
\arg\max_x V(x), x\in F(q)
\end{equation}
Where $x$ is a solution in the solution space $F(q)$, $V$ is the value function specified by animators. 
The function $V$ in practice depends on the application requirement.
For data-driven methods, $V$ may be designed to choose the sequence with most smooth transition.
For kinematic methods, $V$ is designed to select the posture that least violates position constraints\citep{boulic1996hierarchical}. 
With some special energy cost function $V$ and constraints, kinematic methods have been used to retarget motion data\citep{Gleicher1998,Gleicher1998a}.

For dynamic methods, a reasonable value function $V$ is the energy cost. 
\begin{equation}
 \textbf{V}=\int_{t0}^{t1}F_{a}(x)^2dt
\end{equation}
where $F_{a}$ is the active force generated by actuators like motors or muscles. 
This is introduced to CMS research as the influential Spacetime Constraints\citep{Witkin1988}. 
It is based on the hypothesis that the natural looking trajectory costs minimum energy. 
It is related to the idea of Darwin's Theory of Evolution and the principle of Natural Selection. 
In many cases, these methods produced very believable motions. 
\citet{Jain2009} provides an example of locomotion;  
\citet{BalanceControl} find a method for balance maintaining movement. 
\citet{Liu2009} proposed a method for object manipulating animation. 

One shortcoming of Spacetime Constraint is the efficiency.
Spacetime Constraint in nature is a variational optimization problem. 
It takes prohibitive long time to simulate complex musculoskeletal structure\citep{Anderson2001}. 
Optimization techniques like time window and multi-grid techniques are proposed by \citet{Cohen1992} and \citet{Liu1994}. 
Finally, \citet{Popovi'c1999} proposed a method based on Spacetime Constraint for full body dynamic animation.

\section{Limitations of Trajectory based Model}
\label{sec:limitation}
All the methods mentioned in the previous section are based on the same motion model.
In practice, they usually work together. 
For example, a reaching motion can be generated in the following steps: 
(1) motion capture data are selected as the motion reference; 
(2) Inverse Kinematic methods are used to fix the endpoint error; 
(3) Inverse dynamic and optimization are applied to determine the active forces of muscles.
(4) Forwarding simulation is carried out to add responsive effect for perturbations. 
However the methods based on trajectory model are still suffering from the following problems.
\begin{description}
\item [Computational Complexity]
There is no efficient numeric method for variational optimization like Spacetime Constraint. 
Current numeric methods are very sensitive to model accuracy and initial conditions, and converge slowly. 
In many cases, it will take very long time and there is no guarantee the optimal solution can be achieved.
\item [Over Specific]
Most controllers built are only capable of generating animation for a specific motion task under a specific environment condition. 
Current methods only cover a small portion of motion repertoire.
Motions like heart beating, breathing, or motions of other animals such as the swimming of fish and jellyfish, flying of birds have not been synthesized with dynamic methods. 
There is no a general way to model the body and the environment effects. 
It seems that endless controllers need to be designed to tackle all  the combinations of motions and environment.
\item [Artefacts in Low Energy Motion]
\citet{Liu2005} points out that spacetime constraint methods only suit high energy motions like jumping and running.
The results for low energy motion tasks like walking don't look natural. 
The reason is well understood in biomechanics research. 
Muscles, an elastic structure, can store energy during collision impact and release the energy for activating bodies later. 
For low energy motions, this elastic effect can help to save a large portion (as much as 40\%) of energy. 
However because of its complex nonlinear characteristics and large number of muscles involved, this elastic effect is usually ignored in CMS research, mainly for the computation cost reason.
\end{description}
In our view, 
these are serious limitation and hinder the progress of CMS research in many ways. 
Animations for more detailed anatomical structures, such as muscles and tendons will result in prohibitive long computational time. 
Expanding motion repertoire will face the difficulties of modeling fluid and elastic dynamic behaviour and much more computational load.
Following this idea,  
many common daily motion tasks like pouring water into a cup, will be formidable difficult to synthesize dynamically, 
while it is an easy task for human.  
All these problems, in our view come from the motion model. 
It can not be overcome by improving  algorithms. 
Further progress of CMS needs a different motion model.

\section{Methods based on Different Models}
Some CMS research produced good motion results, but have different model for motion.
\begin{description}
\item [Motion Signal Processing]
Motion signal processing(MSP)\citep{bruderlin1995motion} transformed the motion data into frequency domain and introduced the signal processing techniques to CMS research.
MSP is example based and thus shares the disadvantages of other data-driven methods.  
A special advantage of this method is the ability to generating emotion expressive motions. 

MSP may shed light on the motion perception puzzle. 
Humans have a great ability in frequency analysis; the evidences are our sensitivity to colour and musical sound. 
A bold hypothesis we suggest is that maybe the motion perception and even motion control closely relates to our frequency perception power.
\item [Limit Circle]
Limit Circle Control(LCC) \citep{Laszlo1996} provides an alternative method for lower energy locomotion animation. 
The LCC theory has been used in explaining passive mechanics.  
Compared with Spacetime optimization, LLC methods is more computational efficient method for low energy motion.

However in our perspective, the current LLC method has not exhibited its full potential power in theory. 
In current researches\citep{Coros2009,Laszlo1996}, the limited circle is fixed and specified by the animators. 
The control strategies are simplified as a state machine controller following a predefined limited circle.
When Limit Circle is fixed, limited circle control falls into trajectory based methods. 
A prefixed limited circle can be seen as a predefined motion curve of different parameters and value function. 
Fixed limited circle deprives limit circle control of adaptive motion under perturbance.

 
\item [Evolution and Neural Network Methods]
Inspired by the Theory of Evolution and Neural Network, some researches\citep{Sims} build a simple biology system and simulate the evolution process. 
After enough trial and error, reasonable motion controller can be developed. 
But the result is unpredictable. 
Animators have no way to tune the animation.
An impressive idea in this research is not only the controller but also the body structure should be evolved.
\end{description}

Our research started from improving the MSP and LLC methods.
The research idea is to make the LLC adaptive with perturbations. 
In Qualitative Control Theory, we propose that the limit circle is a qualitative property of the dynamic system.
It is also determined by the topological structure and can be qualitatively controlled.
Another research idea is to enhance MSP method with dynamic motion synthesis capacity. 
Although our research is biology grounded and involves neural activity modeling, 
it is not an artificial neural network based method, 
in qualitative control theory, 
we model the dynamic behavior of neural signal and utilize the qualitative dynamic properties. 
From the evolution based research, we don't use such computational method, 
but we accept the idea that the role of body structure is important in motion.







