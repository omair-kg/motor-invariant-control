\section{Mechanical Coupling and Secondary Motion}
Neural Oscilator and Symmetry Control can provided a principle for motor control.
While for mechanical system, like human, usually, the dof is much larger than your application examples.
Given the waking example, besides the two legs, there are also the torsqo,arm and swaying and yawling dof.
there are two basic method for incooprate more dofs.

(1) develop the dynamic system for the fullbody and try to work out the symetry.
(2) the extra dof are not controlled and freely influenced by the controlled dof through mechanical coupling.


for animation application purpose,
we use the second one.


\subsection{mechanical coupling}
for a dynamic system $X=[q_1,\dot{q_1},q_2,\dot{q_2}]$
described by the differential equation.
\[
\dot{X}=F(X)
\]
can be write in a different manner
\[
\dot{x_1}=F_1(x_1)+C_1(x_1,x_2)
\]
\[
\dot{x_2}=F_2(x_2)+C_2(x_1,x_2)
\]
\[
x_1=[q_1,\dot{q_1}]
\]
\[
x_2=[q_2,\dot{q_2)}]
\]

if $C_{1,2}<<f_{1,2}$
dynamic property is domonited by $f$, and $c$ can be treated just perturbation.


for the walking example, the knee, arm and torso, have little effect on the basic walking motion,
we can add sidesway and torso to the orignal walker,
we found the basic motion pattern is kept, only a small variation to the original model.

figure bellown compare the motion include the sidesway and torso.


