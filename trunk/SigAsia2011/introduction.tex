\section{Introduction}
It is important to know the things being animated when we develop animation system. 
Physically based character motor synthesis includes modelling both the body mechanics and the neural system behaviour. 
While techniques for simulating mechanics of body is sophisticated, little is known about the biological motor control principles. 
In mathematical view port, the key problem in motion synthesis comes from the redundant degrees of freedom.  
For a motion task like picking up an apple, human has many different ways to finish it. 
Many biology researches suggested that natural motion is  energy efficient, thus current Researches follow the pioneering work by Wikin(spacetime constraint), formulating the motion synthesis as constrained variational optimization. 
However, space-time constraint entails large computational burden and sensitive to errors, makes it highly unlikely the paradigm for biological motor control. 
Besides energy efficient result, biological motor control also has to be agile, adaptive. 
After putting weight on shoulder, human adjust his gait immediately, without thinking for long.  
Also it is highly unlikely that human detect motion artefacts in this manner, human detects motion artefacts immediately, without long computation.

In this paper, we propose a motion synthesis framework based on a different biological motor control paradigm. 
Some Biology Research suggested that motor repertoires are composed of a number of basic elements, motion primitives. 
Neural system tweaks the motion primitives for environment constraint or special motion purpose. 
Complicate motions are formed by connecting different motion primitives together, just like combining alphabet into sentences. 
The new framework can generate adaptive and energy efficient motion with low computational cost. 



The mathematical framework we developed is based on group action and invariants. 
New motion $m_n$is treated as an element $m$  that modified by a group action $g$ formulated as Equation \eqref{eq:mgm})
\begin{equation}
	m_n=g(m)
\label{eq:mgm}
\end{equation}
	 										
The important property of $g$  is that some motor features are kept invariant, which are motor invariants  $I$. 
Actions $g$ and $m$  keeps motor invariant form the action space $G$ and motion space $M$ as Equation (2). 
To synthesize target motion $m_n$ , we need to find $g  \mid m_n=g(m)$ .
\begin{equation}
	I(g(m))=I(m),g \in G, m \in M
\end{equation}								

For animation purpose, $I$ should contain features that natural looking. 
A biological hypothesis is that when $I$ is violated, human will detect artefacts in motion. 
Mathematics provides some deeper information about $I$.
 For physically based animation, m is the solution of a differential equation that described the dynamics. 
In this paper, we propose two important motor invariants. 
The global motor invariant determines the qualitative properties and local motor invariant which determines the details of motion. 
In mathematical view, they are the Topology and Lie Group Symmetry of the corresponding differential equations that describe the dynamics of motion.

\subsection*{contribution}

\textit{Adaptive}
This framework solves the stability controls and retargeting problem in a unified manner, thus provide more types of adaptation. 
Traditionally, for walking example, for impulse perturbation, terrain, and crippling leg, traditional we need different controllers, while our method can generate all the adaptive motions for all conditions above.

\textit{Computational Efficient}
Finding element in and group involves much less computational work than optimization.
The computational load of our method is extremely low; calculation only involves close form calculation. 
Compared with mechanical simulation, the computational load can even be neglected.

\textit{Artists Directable}
Usually, physically based animation, it is not very intuitive to modify control parameters modify motion. T
he control parameters of our method are more intuitive use. For walking motion, we include some parameters like walking speed, step size or upslope angle.

\subsection*{outline}
After a quick review of physically-based character motion synthesis in Section 2.1,  we also reviewed some important research related to our work on biological motor control and Mechanical Symmetry.
In Section 3 investigate the global motor invariant, details about its effects on motion stability and adaptation. A biological based method for keep global motor invariant is presented in section 3.4
Section 4 focuses on local motor invariant; we propose two groups that modifies the space and time properties of motion independently. A computation efficient method for control is presented in Section 4.3
Section 5 focus on combined the methods maintain global and local motor invariants together.
Application and Motion Synthesis result in presented in section 5 and Section 6 discuss some features of this method and further research work.

