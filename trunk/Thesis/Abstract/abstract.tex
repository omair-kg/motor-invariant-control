
% Thesis Abstract -----------------------------------------------------


%\begin{abstractslong}    %uncommenting this line, gives a different abstract heading
\begin{abstracts}        %this creates the heading for the abstract page

Generating natural-looking motions for virtual characters is a challenging research topic.
It is even harder to synthesize adaptive motions interacting with the environment. 
Current methods are tedious to use, cost long computational time and fail to capture natural looking features.

In this thesis, we propose an efficient method of generating natural-looking motion based upon recent advances in biology research.
The principal idea is motor repertoire is made up of a limited number of elements called motion primitives. 
Motor control basically connect the basic motion primitives together just like connecting alphabets into sentences.


Motion primitives serve as templates, and neural system tweaks basic templates to generate motion satisfy constraints of environment or motion purpose.
In this process, some properties of motion primitives are maintained, which are called motor invariants.

%we propose principle of motor control is not feedback based, they should by model as topology conjugacy,mechanical system to form an analogous dynamic system that meets constraints and purpose.

We introduce new mathematical tools for identify and tweak motion primitives.
Motor invariants are identified by ideas of equivalence and symmetry.
Motion Primitives are identified by the qualitative properties, which is captured by the differential topology.
Motion adaptation are modelled as topological conjugacy,the tweakings maintain the topology and result in analogous systems
Neural Ocillator and Symmetry Preserving transformations provide computational efficient methods for such purpose.

Based on the new theory, we generate adaptive motions in real-time while maintaining the natural looking figures.

\end{abstracts}
%\end{abstractlongs}


% ----------------------------------------------------------------------


%%% Local Variables: 
%%% mode: latex
%%% TeX-master: "../thesis"
%%% End: 

