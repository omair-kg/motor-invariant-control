
% Thesis Abstract -----------------------------------------------------


%\begin{abstractslong}    %uncommenting this line, gives a different abstract heading
\begin{abstracts}        %this creates the heading for the abstract page

Generating natural-looking motions for virtual characters is a challenging research topic.
It becomes even harder when generating adaptive motions interacting with the environment. 
Current methods are tedious, cost long computational time and fail to capture natural looking features.

This report proposes an efficient method of generating natural-looking motion based upon a new motor control theory.
The principal idea is motor repertoire is made up of a limited number of elements. Motor control basically connect the basic motion primitives together just like connecting alphabets into sentences.


Motion Primitives are identified by the qualitative properties, for which we use the mathematical tools of differential topology.
Tweaking of the motion primitives is model as Symmetry Preserved Transformation, for which we use lie group theory.


Our method can generate adaptive natural looking motion with very little computational costs.

\note{Topology Conjugacy}
\note{Symmetry shoud ref discrete symmetry writing}
\note{System Affine Transformation}
\note{Kangroo example? how to add leg swing? Running how to switch leg?}
\note{appendiex mathematical}

\end{abstracts}
%\end{abstractlongs}


% ----------------------------------------------------------------------


%%% Local Variables: 
%%% mode: latex
%%% TeX-master: "../thesis"
%%% End: 
