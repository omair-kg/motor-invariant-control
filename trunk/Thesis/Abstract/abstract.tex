
% Thesis Abstract -----------------------------------------------------


%\begin{abstractslong}    %uncommenting this line, gives a different abstract heading
\begin{abstracts}        %this creates the heading for the abstract page

Generating natural-looking motions for virtual characters is a challenging research topic, made even harder when adapting synthesized motions to interact with the environment. 
Current methods are tedious to use, computational expensive and fail to capture natural looking features.
These difficulties seem to suggest that artificial control techniques are inferior to the natural counterparts.

Recent advances in biology research point to a new motor control principle: utilize the natural dynamics.
The interaction of body and environment form some patterns, which works as primary elements for the motion repertoire: Motion Primitives
These elements serve as templates, tweaked by the neural system to satisfy  environmental constraints or motion purpose.
Complex motions are synthesized by connecting motion primitives together, just like connecting alphabets into sentences.


%we propose principle of motor control is not feedback based, they should by model as topology conjugacy,mechanical system to form an analogous dynamic system that meets constraints and purpose.

Based on such ideas,   this thesis proposes a new dynamic motion synthesis method.
A contribution is the insight of dynamic reason behind motion primitives: template motions are stable and energy efficient. 
When synthesizing motions from templates, valuable properties  like stability and efficiency should be well preserved.
The mathematical formalization of this idea is the \emph{Motor Invariant Theory} and the preserved properties are \emph{motor invariant}

In the process of conceptualization, new mathematical tools are introduced to the research topic.
The Invariant Theory, especially mathematical concepts of equivalence and symmetry, play the crucial roles.
Motion adaptation is mathematically modelled as topological conjugacy: a transformation which maintains the topology and results in an analogous system.

The \emph{Neural Oscillator} and \emph{Symmetry Preserving Transformations} are proposed for their computational efficiency.
Even without reference motion data, this approach produces natural looking motion in real-time.
Also the new motor invariant theory might  shed light into the long time perception problem in biological research.

\end{abstracts}
%\end{abstractlongs}


% ----------------------------------------------------------------------


%%% Local Variables: 
%%% mode: latex
%%% TeX-master: "../thesis"
%%% End: 

