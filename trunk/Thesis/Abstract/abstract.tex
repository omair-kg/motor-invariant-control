
% Thesis Abstract -----------------------------------------------------


%\begin{abstractslong}    %uncommenting this line, gives a different abstract heading
\begin{abstracts}        %this creates the heading for the abstract page

Generating natural-looking motions for virtual characters is a challenging research topic. 
It becomes even harder when generating adaptive motions interacting with the environment. 
Current methods are tedious, cost long computational time and fail to capture natural looking features.

This report proposes an efficient method of generating natural-looking motion based upon a new motion control theory.
We propose that \textbf{only the qualitative properties of motion are controlled, adaptations are generated by perturbations of environment or body structure}, this is the  \textbf{the Qualitative Control Theory (QCT)}.
Inspirations for this research come from the contradiction between biological facts and current motion synthesis ideas. 
The biological idea we hold  is that natural-looking motions mainly come from the complex interaction between the body and the environment. 
The natural neural system only maintains or tweaks qualitative properties of this dynamic interaction. 
We believe that motion is composed of many motion primitives, each motion primitive is a \textbf{structural stable autonomous system}.
Motion and Adaptation can be generated without any control effort.
The mathematical model we propose for the natural motion is based on the Qualitative Theory of Differential Equation.
Qualitative Control is achieved through manipulating the topological structure of the dynamic system to enhance the `` self-balance'' ability, rather than counteracting the perturbation effects. 
The control method we propose is well supported by biology research.

Adaptive Motion Synthesis Method completely solving  motion retargeting problem.
Motion data can be directed applied for a different character in a different environment.
Also this method involves with very light computational load.
Further, it maintains important features of natural looking motions.

Based on such results, 
we discuss potential applications of this method to full body character animation synthesis,
which forms the main job of the PhD phase research.



Physically Based Motion Synthesis is challenging, mainly because of Freedom Curse, which usually involves lots of computational work. 
In this paper, we develop a new method based on the biological idea of motion primitive. 
Motion Synthesis and Motion Retargeting are achieved by modifying the basic motion primitives. 
Our Method separate stability control and motion style in two decoupled problem. 
In mathematical viewport,  the method control the topology and space of flow independently.
Our method can generate adaptive motion, is applicable for a large number of system and is computational efficient.


\end{abstracts}
%\end{abstractlongs}


% ----------------------------------------------------------------------


%%% Local Variables: 
%%% mode: latex
%%% TeX-master: "../thesis"
%%% End: 
