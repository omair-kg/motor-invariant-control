\chapter{BACKGROUND}
\label{chap:background}

\nomenclature[z23]{\pd}{Proportional Derivative}
\nomenclature[z24]{\lc}{Limit Cycle}
\nomenclature[z25]{\cpg}{Central Pattern Generator}
\nomenclature[z26]{\eph}{Equilibrium Point Hypothesis}
\nomenclature[z27]{\umh}{Uncontrolled Manifold Hypothesis}
What differ \cms methods are ideas of motor control.
Many current \cms research studies adopted the control hierarchy of artificial systems.
No matter whether the control method is based on tracking or optimization, in such systems, there is a clear separation of planning and execution.
Body are treated as mechanical apparatus, which execute the motor commands from the neural system.

Motor Invariant Theory(\moit) is based on the integrative theory of motor control\citep{dickinson2000animals}:
For biological systems, planning and execution can not be separated clearly.
Neural system plays a limited role in the planning.
Body and environment are taken into consideration and motor control can only be understood from a broader perspective.

In this chapter, limitations of current \cms methods are discussed first, which are the motivation of this research.
New theory is developed because these limitations can not be overcome without breaking the already theoretical framework.
Supporting biological research studies are discussed later,  which serve as justification for \moit.



\section{A survey of \cms}

Many methods are developed in \cms, making it impossible to include all the work in this chapter.
For a short discussion,\cms methods are categorized by the principal control model: memory based or computation based.
Memory based control model is foundation of the  many data-driven techniques;
most procedure methods are computation based.
Pros and cons are discussed in categories.

\subsection{Data Driven}
Data-driven methods are based on ready motion data,  generated by Key-frame or Motion Capture(Mocap). 
In practice, motion data are segmented into short time clips. 
An animation is synthesized by selecting motion clips and connecting them together\citep{Parent2002,kovar2003flexible}.

Like other example based methods, data driven methods can generate good results if similar motion clips are available, but difficult to generate  adaptation or novel motion, either for a different character or scenario. 
The  ``re-targeting'' problem is a big challenge.

In practise, motion versatility requires a large data base,  data management becomes another problem. 
The Annotation Database \citep{Arikan2003} and the Motion Graph \citep{kovar2008motion} are proposed. 
Currently, catalogue and search of motion data are not trivial and remain open\citep{keogh2004indexing,muller2005efficient}.

\subsection{Procedural Method}
For physics based \cms, different procedural approaches have been proposed.
\begin{itemize}
\HiItem{Tracking Controllers}




Some early research applied classical \pd controller \citep{Raibert1991} for dynamic motion synthesis.
Later research \citep{Hodgins1995} applied the same method for different tasks like running, bicycling, vaulting and balancing. 
For high dimensional characters, \pd controller tracks the predefined motion curves\citep{Yin2007}.

\pd controller is shown in Equation~\ref{eq:pdcontrol}.
\begin{equation}
\label{eq:pdcontrol}
u=K(q -q_d)+d\qd
\end{equation}
where $u$ is the control effort, $K$ is the stiffness, $q_d$ is the desired or reference position, and $d$ is the damping efficient.
\pd based methods can run in real-time and generate adaptive responses to small perturbation.
But large perturbation response or deviation from the reference trajectory are difficult to achieve.


Most \pd based controllers use motion capture data as references.
As an alternative, \citet{Laszlo1996} introduced Limit Cycle (\lc) as tracking reference for periodic locomotion animation. 
Current research studies\citep{Coros2009,coros2010generalized,Laszlo1996} track fixed limit cycle,
such methods share many characteristics with \pd, which promise real-time speed, but lacks adaptation ability and results stereotype looking motion.







\HiItem{Optimization}
The redundant \dof s make motion planning non-deterministic.
Optimization has been introduced to \cms for this problem.
The idea is to choose the ``best'' one among all the possible motions.

Many merits have been proposed for \cms.
For dynamic methods, a reasonable merit is the energy cost~$E$. 
\begin{equation}
 \textbf{E}=\int_{t_0}^{t_1}f_{a}(t)^2dt
\end{equation}
where $f_{a}$ is the active force generated by actuators like motors or muscles. 
This is introduced to \cms research as the influential Spacetime Constraints\citep{Witkin1988}. 
It is based on the hypothesis that the natural looking trajectory costs minimum energy 
and closely relates to the idea of Darwin's Theory of Evolution. 

Optimization based  methods produced believable motions for variable tasks. 
\citet{Jain2009} provided an example of locomotion.  
\citet{BalanceControl} found a method for balance maintaining movement. 
\citet{Liu2009} proposed a method for object manipulating animation. 
\end{itemize}


\subsubsection*{Drawbacks of Optimization}
Optimization is the current mainstream method for physics based animation.
It generated the best motion results in current research.
But this method has several drawbacks.

\begin{itemize}
\HiItem{Numerical Instability and Modelling Difficulties: }
Optimization methods promise the energy efficiency of the resulting motion, but no guarantee about convergence speed and stability.
Optimal solution is difficult to find numerically, and sensitive to the accuracy of the model and the proximity of the initial guess.
\citet{Liu2005} points out that the primitive space–time constraint methods only suit high energy motions, like jumping and running.
For low energy tasks (such as walking) the results do not look natural.

\HiItem{Computational Complexity: }
Optimization with space–time constraints is a variational problem by nature. 
For a complex character, it might takes  prohibitively long time, limiting the application domain of problems to those which are computationally feasible. 
In addition, little is known about how to reuse a computation result for motion adaptation.
\end{itemize}


\subsection{Fix Up}
There are many research attempts to make tracking controllers more adaptive or optimization faster.
One idea is mix the two methods, optimization is done offline for planning the reference trajectory, while tracking controllers are adopted as online controllers.
Many methods start to train the controller with motion capture data \citep{levine2011space,coros2010generalized,de2010feature,wei2011physically,ye2010optimal,lee2010motion,wang2010optimizing,wu2010terrain,liu2010sampling,lee2010data}, or use simplified model for planning\citep{mordatch2010robust}.


These attempts may ease the limitations of tracking or optimization, and make them feasible for certain applications.
But \cms problems can not be solved completely in this manner.
Learning based methods are complex and  sensitive to training examples.
Further validity of controllers can not be strictly proved.
Besides  offline optimization  does not reduce the computational burden in nature.




\subsection{Biological Constraints}
The problems of \cms has also been spotted earlier in biological motor control research.
Biological researchers have dropped  tradition artificial control ideas long ago, because they  violates the biological constraints.
Although the mechanism behind information processing remains obscure, some characteristics of biological information processing are well agreed,  making  \cms methods above questionable\citep{Glynn2003}. 
  
\begin{itemize}
\HiItem{Sensing and Control Limitations:}
Motor control is not only a mechanical problem, but also a complex process involving chemical, electrical and mechanical changes.
Many crucial mechanical parameters and variables such as mass, inertia, force, are inaccessible to the neural system and can only be approximated. 
Some important control variables (such as torque) are controlled indirectly by the neural system through a complex process.
Also body and environmental measurements are noisy and time varying, making methods sensitive to errors unsuitable for biological motor control.

\HiItem{Neural Computation: }
The neural system is powerful, but inferior in speed and accuracy when compared with digital computers. 
Neural signals are of only hundreds of Hz and their transmission speed is slow.
In addition there is a long delay between firing a neural signal and generating force in the muscles.
It may cost about half a second from seeing an object to force generation in arm. 
This makes it impossible for the neural system to carry out the complex computation necessary for real–time optimization.


Following the idea of optimization control, the dynamics of fluid environment and deformable body are more difficult to optimize. 
But most primitive life forms live in the sea and have limited intelligence. 
\HiItem{Memory Capacity:}
Some argue that motion control is not based on computation, but based on memory.
This avoids the question of computation speed, but it faces the memory capacity problem. 
Motion varies greatly, if we store the motion in our brain.
The capacity problem will arise.
\end{itemize}

Because of such constraints, researchers have started to look for different strategies.

\section{Motion Primitives}
At first, we should remember that logical think or mental conscious plays little role in motor planning.
Animals including human exhibit complex motion behaviours after birth or at early ages, abilities like breathing, heat beating and child bearing are inborn without learning.

Some suggests that motor ability are inborn and organized in blocks\citep{bizzi1995modular,bizzi2002book}.
Strong evidences come from the experiment where stimulating of a single spinal motor afferent triggers a complete sweeping motion\citep{bizzi1995modular}.
A new theory, Motion Primitive Conjecture, was proposed.
In this theory, the building blocks are called \emph{motion primitives}.
Also the theory believes that the number of motion primitives is limited and
complex motions are combinations of motion primitives, just like we connect alphabets into sentences.

Motion Primitive Conjecture also provides insight into the motion perception.
\citet{gallese1996action} have found action and perception trigger similar reactions in a group of neurons.





\subsection{Dynamic Motion Primitives}
The Conjecture of motion primitive is supported by both the behaviour study and anatomy of natural animals.
For dynamic \cms, the puzzle is how motion primitives simplify dynamic motor control.

An proposed answer is that motion primitives have some valuable dynamic properties, like stability and efficiency, and are based on natural dynamics.
Some researchers point out that motion is closely related to the body structure and environment,
and not changed much by the evolution of neural system, after all the whales swim more like fish than other mammals.
Animals do not move the way they want, but rather the way they can. 
The body and the environment play the most important role in motor control, as they form the basic pattern of motion \citep{nishikawa2007neuromechanics}.
For neural control, the responsibility is not planing the trajectory from ground up, but modifying or tweaking basic patterns that meet the purpose constraint without losing the crucial properties.
Several theories are proposed for the neural control mechanism.

Experiments have shown that even under the same conditions, the motions still vary. 
Some \dof s are not controlled and freely influenced by the environment. 
\emph{Uncontrolled Manifold Hypothesis(\umh)}\citep{latash2008neurophysiological} proposes in motor control, trajectory is not the concern, only the final results is.


\emph{Equilibrium Point Hypothesis(\eph)}\citep{feldman1986ome} is a specification of \umh.
This idea comes from properties of differential equations. 
For a dynamic system
\[
\dot{\state}=F(\state)
\]
the equilibrium points $\state_{e}$ satisfy the condition $F(\state_{e})=0$.
\eph suggests the neural system does not plan motion trajectory, but the position of the equilibrium point.



\emph{Impedance Control} \citep{hogan1985ica} refines the idea of \eph by providing an explanation for effects of the extra \dof s. 
At an equilibrium point $\state_{e}$,
\[
F(\state_{e})=0 
\]
Impedance Control proposed that the extra \dof s provide a way to control the stability and admittance of the equilibrium point $\state_{e}$. 
The mathematical presentation is
\begin{equation}
F(\state_{e}+E_r)=KE_r
\end{equation}
where $E_r$ is the offset error vector, $K$ is stiffness matrix or impedance,which determines the stability.
Neural system will tune the direction of $K$ according to the purpose,  avoiding obstacles or risks. 
Experiments \citep{Franklin2007} have shown  that the matrix $K$ has anisotropic properties.







\subsection{ Neural Control Mechanism}
Motor control involves little mental work,
and current idea of neural science is that motor control is a low level intelligent activity and can be controlled  without brain input. 
Research studies have proposed several neural activities related to its role in ``tweak''  motion primitives.
\begin{itemize}
\item
In vertebrate animals,  Central Pattern Generator (\cpg) serves important functions in locomotion, respiration, swallowing and other rhythm behaviour.
\citet{Cohen1988a} argues that locomotion is the result of the interaction between neural and mechanical oscillators via a process called \textbf{entrainment}.
Neural systems modify the motion by adjust frequency and amplitude of neural rhythmic signal.



\item
Some research studies find out that motion will change in a uniform manner\citep{Viviani1992},\citet{flash2007affine} propose modelling motion adaptation through \emph{affine transformation}.
Motion Adaptation can be model as group action.
What makes the theory more interesting is the fact that affine transformation group is closely related to the vision system, which implies a close relationship between motor control and vision.

\end{itemize}









\subsection{Bionic Robotic Research}
Ideas from biological research  also inspired robotic engineering experiments,which show the feasibility of new control principles.
Such robots utilize the natural dynamic rather than the tracking or optimization strategy.
Here are some reported researches.
\begin{itemize}
\HiItem{Limit Cycle in Walking}
A very important discovery is the bipedal walking can happen without any control\citep{McGeer1990}.  
Under specific condition, a mechanic structure can walk down a slope passively, with natural looking gaits.
And based on this idea, new mechanical system is designed that can walk on plane with simple control strategy\citep{Collins2005}.

\HiItem{\cpg and entrainment}
The \cpg based entrainment is applied for robotic research\citep{Williamson1999a}, the found results show the \cpg will boost the system stability and can maintain motion in unpredictable situation.
\citet{fukuoka2003adaptive} has applied \cpg for quadrupedal walking.

\HiItem{Passive based Control}
The control and mechanics community also starts thinking about passive based control method that utilize the natural dynamics.
Controlled Lagrange \citep{bloch2000controlled,bloch2001controlled,bloch2000potential}and Controlled Symmetry \citep{spong1998underactuated,Spong96energybased} are mathematical rigid method and have been applied to mechanic control and robotics\citep{spong2005controlled}.
\end{itemize}

\section{Placement and Contrasts}
New biological theories become the foundation of \moit.
 \moit introduces topological conjugacy as the foundational modelling theory  and unifies different ideas from a new  perspective.
In \moit, Motion Primitives are identified by their \emph{structural stability}.
the principle of \eph and Impedance Control are generalized as attractor and attraction control.

\cpg comes from the research of spinal cord, which models the low level control; while the transformation idea comes from research of the cortex, which is a model for high level control.
At first sight, Entrainment of \cpg and Group Transformation seem very different in principle , but  in \moit, both control principles maintain the qualitative property of natural dynamics.
Such unification implies a new control hierarchy framework and has a good biological meaning: 
The low level control utilize some robust and qualitative measures like entrainment or \cpg to boost the stability of motion;
while high level control is quantitative and  precise, which adapts stable motion for specific purpose.

\moit also contributes ideas about  motion primitives transitions.


 

 



