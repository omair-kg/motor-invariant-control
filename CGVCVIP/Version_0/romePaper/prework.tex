\section{Related Work}

\subsection{Dynamic CMS Methods}
For procedural methods, motions are formulated as functions of various parameters such as
\[
F:P \mapsto M
\],
where $M$ is the valid motion space of characters, $P$ is the control parameter space.
Currently, $P$ and $F$ are mainly based on rules of physical science, and such procedural methods are often called ``Simulation Methods''. 
Since natural motion is governed by mechanics, dynamic simulation has the potential to generate more realistic motions. 
For dynamic methods, the body is usually modelled as a linked rigid body system,and simulated by realtime methods\citep{Baraff1994,Mirtich1996,Stewart2000}. 
The bodies of animals are actuated by muscles under the control of neural system. 
Control system design is the most difficult problem.

Some early research applied classical control methods like PD controller \citep{Raibert1991} for locomotion synthesis.
Later research \citep{Hodgins1995} applied the same method for different tasks like running, bicycling, vaulting and balancing. 
Such methods are based on simplified models which relieved the controllers from the problem of redundant DOFs, 
but important motion details were also neglected.


In most cases, motion solution are not unique.
Optimization methods have been applied to solve the nondeterministic problem. 
Among all the solutions in possible motion space, the ``best'' one is chosen as the proper solution:
For dynamic methods,  a reasonal choice is minimize the energy cost~$V$,such that 
\[
\textbf{V}=\int_{t0}^{t1}F_{a}(x)^2dt
\]
where $F_{a}$ is the active force generated by actuators like motors or muscles. 
This is introduced to CMS research as the influential Spacetime Constraints\citep{Witkin1988}. 
In many cases, these methods produced very believable motions. 
\citet{Jain2009} provides an example of locomotion;  
\citet{BalanceControl} find a method for balance maintaining movement. 
\citet{Liu2009} proposed a method for object manipulating animation. 
One shortcoming of Spacetime Constraint is the efficiency.
Spacetime Constraint in nature is a variational optimization problem. 
It takes prohibitive long time to simulate complex musculoskeletal structure\citep{Anderson2001}. 
Optimization techniques like time window and multi-grid techniques are proposed by \citet{Cohen1992} and \citet{Liu1994}. 
Very a few research \citep{Popovi'c1999} proposed Spacetime Constraint for full body dynamic animation.

Limit Circle Control(LCC) \citep{Laszlo1996} provides an alternative method for lower energy locomotion animation. 
The LCC theory has been used in explaining passive mechanics.  
Compared with Spacetime optimization, LLC methods is more computational efficient method for low energy motion.

Inspired by the Theory of Evolution and Neural Network, some researches\citep{Sims} build a simple biology system and simulate the evolution process of brain. 
After enough trial and error, reasonable motion controller can be developed. 
But the result is unpredictable. 
\subsection{Biological Research}
The foundation of Motion Synthesis is our understanding of natural animals' motor control system. 
In biological viewport, motor control is an age old problem full of paradoxes.
Motor Control is a complex process involves many chemical, electrical and mechanical effects.
In both CMS and biological motor control research, one most noticeable question is the computational efficiency.
More questions arise after more knowledge of the biological computer, the neural system, has been obtained,
which makes traditional control idea questionable. 
Here we list several major questions\citep{Glynn2003}.  

\textbf{Time Delay}
Neural signal transmitting speed is slow; and there is a long delay between neural signal firing and force generation in muscles. 

\textbf{Noisy}
Besides the delay and slowness, the neural signals are also noisy. 
The body structure and environment are also nonlinear, noisy and time varying. 
So methods that are sensitive to model accuracy are not proper for the natural neural control system.

\textbf{Limited Activity}
Current research evidences and common life experience show that motor control involves little control effort. 
Many experiments show motion can happen even without brain input. 


Despite the complexity of body structures and environment, the natural motor control strategy seems relatively simple, involves little computational work. 
The current idea of biology research is that motor control is a low level intelligent activity and can be controlled with some very primitive neural structure. 
In many animals, the active neural structure in motor control is the Central Pattern Generator(CPG) which generates rhythmic signals.
There are many experimental researches in robotics and biomechanics succeeded in controlling some motion with very simple strategy\citep{nishikawa2007neuromechanics}.
And some new ideas about motor control have formed.

\textbf{Uncontrolled Manifold Hypothesis}
The observation of blacksmith's hammering motions show that even under the same conditions, the motions still vary. 
An explanation is the neutral system doesn't control all the DOFs. 
Some DOFs are not controlled and freely influenced by the environment. 
This is the Uncontrolled Manifold Hypothesis(UMH)\citep{latash2008neurophysiological}. 
In this viewpoint, the result of motion planning is not a trajectory, but a space of valid trajectories.

\textbf{Equilibrium Point Hypothesis}
EPH suggests that what the neural systems controls is not trajectory, but the equilibrium points.
This idea comes from properties of differential equations. 
For a dynamic system in the form $\dot{q}=H(q)$,
the equilibrium points $q_{e}$such that $H(q_{e})=0$.
For a stable system, over the time the state $q$ will approach to the equilibrium point $q_{e}$ and finally stays at $q_{e}$.


\textbf{Impedance Control Hypothesis}
Impedance Control \citep{hogan1985ica} refines the idea of EPH by providing an explanation for effects of the extra DOFs. 
Impedance Control proposed that at an equilibrium point $q_{e}$ such that~$H(q_{e})=0$,
the extra DOFs provide a way to control the stability and admittance of the equilibrium point $q_{e}$. 
The mathematical presentation is
\begin{equation}
H(q_{e}+Er)=K
\end{equation}
where $Er$ is the offset error vector, $K$ is stiffness matrix or impedance, which determines the stability and gentleness of the equlibrium point.
Neural system will tune the direction of $K$ according to the motion purpose, such as avoiding obstacles and risks. 
Experiment \citep{Franklin2007} shows that the matrix $K$ has anisotropic properties.

\textbf{Morphological Computation Theory}
UMH,EPH and IMH are efficient at explaining some arm motion and object manipulation tasks, but the theory is incomplete for more complex motion.
A generalization theory is proposed as Morphological Computation Theory(MCT)\citep{nishikawa2007neuromechanics,Pfeifer2005}.
The idea is both the body structure and the environment play a crucial role in  motor control, 
they can be treated as a physical computer. 
For some motion tasks, basic motion patterns are generated by body and environment,
the neural systems only  maintains or tweaks such motion patterns.
