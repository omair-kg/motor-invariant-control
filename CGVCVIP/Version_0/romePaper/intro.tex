\section{Introduction}

The challenge of Character Motion Synthesis (CMS) is not to make characters move, but how to make them lifelike. 
This comes from our human's marvellous ability of motion perception. 
Motions for the same task are very similar, but vary adaptively.
From the variety in motion details, humans can infer the changes in mental states, health conditions or even the surrounding environment. 

Nowadays in industry, high quality motions are majorly generated by manual work. 
Most characters are very complex, which contain a large number of joints, making animation a tedious work.
To make things worse, it is difficult to reuse animation data. 

Some Researchers hold the belief that motion can be synthesised by simulating the dynamics of body, environment and the neural control system,while it is not an easy task.
From the mechanical viewpoint,  
virtual characters are full of redundant \textbf{degree of freedom} (DOF)s,  which not only increase the computational load, but also make the solution nondeterministic. 



Some features are noticable in bioligical system but are hard to achieve at the same time by current CMS methods. 

\textbf{Adaptive} 
Natural motions are adaptive to the changes in the environment or body conditions. 
A typical example is human locomotion. 
The walking motion changes on different terrains. 

\textbf{Agile}
Some motions of animals are very fast, 
more puzzling is that the neural system can solve the complex motion control problem in a realtime. 
It seems very easy for the neural system to solve such complicate problems.

\textbf{Efficient}
Natural Motions are energy efficient.
In theory, this idea is supported by Darwin's Theory of Evolution.
Animals spent far less energy than our expectation.
An example is that the energy consumed by human walking is only 10\% of that for a robot of the same scale.




In QCT, the objective of motion control is some qualitative property of the motion.
Adaptation involves little  control effort..
In mathematics, a natural motion is modelled as a \textbf{structural stable autonomous system}.
The key idea of Qualitative Control Theory is focusing on the topological structure of such dyanmic system.

For biology theory, the proposed research further completes the theory of \textbf{Motor Control} . 
For CMS research, this research introduces a novel method aiming at generating adaptive motions. 
In application, it will solve the problem of reusing motion data thus greatly reduce the animation work.
It especially suits repetitive and low energy motion tasks which are most challenging to synthesize.
It also has the advantage of computational efficiency and can be accelerated by GPU and may run in real-time in future.


