\section{Background and Previous Work}
\subsection{dynamic simulation}
Dynamic Motion Synthesis  tries to synthesize character motion through physical simulation of the mechanic structure of character body which is usually modelled as a linked rigid body system \cite{Baraff1994,Mirtich1996,Stewart2000}. 
Since many real physical properties are considered in the computation, the generated motion are normally physical feasible. 
However the most difficult task for those methods is to design a efficient control system to simulate the functionality of a real biological neural system. Some early research applied classical control methods like PD controller \cite{Raibert1991} for locomotion. 
Later research \cite{Hodgins1995} applied the same method for different tasks like running, bicycling, vaulting and balancing. 
Limit Circle Control(LCC) \cite{Laszlo1996} provides an alternative method for lower energy locomotion animation. 
However both the classical PD controller and Limit Circle Controller predefined motion trajectories and eliminated perturbations. 
This make them not good at generating motion adaptation.

Because lots of degrees of freedom are involved in the whole body simulation, in most cases, motion solutions are not unique.
Many optimization methods have been applied to choose the ``best'' motion. For dynamic methods, a reasonable choice is to minimize the energy cost~$V$, such that 
\[
\textbf{V}=\int_{t0}^{t1}F_{a}(x)^2dt
\]
where $F_{a}$ is the active force generated by actuators like motors or muscles. 
This is introduced to CMS research as the influential Spacetime Constraints\cite{Witkin1988}, and serve as the foundation for many modern CMS research. \cite{Jain2009} provides an example for locomotion;  \cite{BalanceControl} find a method for balance maintaining movement. \cite{Liu2009} proposed a method for object manipulating animation.

\subsection{Optimizaiton Based Method}
Optimization provided an idea for solving the redundant freedom problem and complies with energy efficient principle of natural motion.  
But this method has several drawbacks for motion synthesis application.

(1) Numeric Stability and Modelling Difficulties
Optimization method only grantee the energy efficiently of the motion, but nothing is about the converging speed and stability. Even the optimal solution is natural looking, it is very hard find.
it can obtain the solution when the model is accurate and the initial guess is near the optimal solution.
So this method is sensitive to model error and initial condition. 
While for motion, the accurate model is very hard to achieve and artefacts are generated.
Liu2005  points out that spacetime constraint methods only suit high energy motions like jumping and running; for low energy motion tasks like walking the result doesn't looks nature. This is mainly because the muscle effects are neglected.

(2) Computational Complexity
Motion Control is variational problem in nature. For complex body structure, even the state of art numeric method will take inhibitive long time. And little is know about how to reuse the computation result for motion adaptation.

(3) Limited Application Domain
Optimization method nowadays only applies for very limited motion. Main for motion can by model with rigid body.
 A Large of motion are still uncovered. Like heart beating, breathing, fish swims and bird flying, such motion involves the soft body and fluid dynamics. Such model are computational difficult in nature, apply optimization based method for such dynamic are not computational feasible.
 
\subsection{motion perception problem}
For motion synthesis research in Graphics, we focus on generating natural looking motion rather than physically-correct motion. 
As long as the audience don’t notice the artefacts, such result will be OK. 
A important biological question is how human recognize motion and detect the artefacts.
 Natural looking motion observed from life must be physically feasible, but it does not mean human detects motion artefacts by doing physical calculation in mind for the speed limitation, nor it is only because memory of motion for the capacity limitation.
\subsection{Limit Circle Based Method}

\subsection{Motion Primitives}
\subsubsection{The Lessons from Biology Research}
Motor Control from biology research is full of paradoxes.
Motor Control in nature is a very complex process, it involves electrical, chemical and mechanical changes. However, for such complex problem, seems an easy task for most human and animals.
From biological view port, the problems of optimization are worse. Natural motor control faces many limitation of the biological motor control system

2.2.1 Biological Constraints
(1) Sensing and Control Limitation
Motor control is not only a mechanical problem, it is a complex process. 
For the biological system,  many crucial parameters and variables are not accessible. 
Dynamic model, force, torque, angle can only sensed with approximation, while mass, inertia, and human have not direct sense. 
For some control variables like toque, neural system has no direct control access.

Besides this, body and environment is noise and time varying.
Methods are sensitive to errors are not suitable for biological motor control.

(2) Neural Computation
The neural system is powerful but it is inferior in speed and accuracy when compared with digital computer. 
Neural signal transmitting speed is slow; and there is a long delay between neural signal firing and force generation in muscles. 
It is impossible for neural system to carry out complex computation for optimization in real-time time.

2.2.2 Biological Discoveries
Limited Computation Activity
Common Life experience shows motor control maybe an easy task. 
This is proved by some biological research. Many experiments show motion can happen even without brain input.
Development
Many motor ability are born, rather than developed. 
Eating, Breeding, Breathing and locomotion, many motor ability in animals seems is inborn rather than learned. 
Evolution
Form Evolution view port, the motion style seems not close related to the development of neural system. 
Wales and Fishes swim in a similar manner.
What is apparent is the body and environment. Animals with a similar body in a similar environment usually move in similar manner.

\subsubsection{ Different Biological Motor Control idea}
Biological Research propose some different idea about motor control, which takes the biological constraints into consideration.
 equilibrium point hypothesis and uncontrolled manifold hypothesis
An idea of motor that will simply the computation and maintain the energy efficiency is eph and ump. Some proposed that neural system doesn’t control all the system,  it may only control the final motion result. As long as the motion finish its task, it don’t matter how it is carried out.
This means control final motion position, and let motion move freely according to the environment and body condition. Which should be computational efficient and energy efficient.
 morphological computation and motion primitives framework.
Animals don’t move the way they want, but only the way they can.
It is the body and environment plays the most important role in motor control, they forms the basic pattern of motion, neural system only tuning or tweak the basic pattern to fit animal’s special needs.
More complex motions are based on combine the basic pattern together.
This idea is easy to compute and can provide a way for motion perception.
Basic pattern are limited and human are very familiar. When the pattern is break, human will notices the difference will result motion artefacts.
2.4 The relationship between our research and new ideas
There is no a unified way to identify motion pattern and motion primitives. In our framework, we based on the idea of group and invariant.
Motion Primitive in our eyes is the global qualitative properties of motion, which is capture by the topology structure of the dynamics of body and environment. This is the Global Motor Invariant.
EPH are adopted as a method for tweaking basic pattern. The method we propose for equilibrium point control is based on symmetry properties of differential equation. This is the Local Motor Invariant.
Because our method is based on dynamics, so it is physically-based method. The advantage of Goup and Invariant theory is that it will provide an efficient computational method.