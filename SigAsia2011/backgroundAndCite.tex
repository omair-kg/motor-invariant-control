\section{Background Work}
physics based motion synthesis is a topic invovles different reseach filed.
\subsection{Motion Control Methods}
\textbf{PD based controller}


Some early research applied classical control methods like PD controller \cite{Raibert1991} for locomotion. 
Later research \cite{Hodgins1995} applied the same method for different tasks like running, bicycling, vaulting and balancing.
Reference motion is needed for PD controller. 
Such controller can genarate responsive motion,but motion is not adaptive.

\textbf{Limit Circle}
\cite{Laszlo1996} provides an alternative method for lower energy locomotion animation. 
Limit Circle Controller predefined motion trajectories and eliminated perturbations. 



\textbf{Optimization}
Because lots of degrees of freedom are involved in the whole body simulation, in most cases, motion solutions are not unique.
Many optimization methods have been applied to choose the ``best'' motion. For dynamic methods, a reasonable choice is to minimize the energy cost~$V$, such that 
\[
\textbf{V}=\int_{t0}^{t1}F_{a}(x)^2dt
\]
where $F_{a}$ is the active force generated by actuators like motors or muscles. 
This is introduced to CMS research as the influential Spacetime Constraints\cite{Witkin1988}, and serve as the foundation for many modern CMS research. \cite{Jain2009} provides an example for locomotion;  \cite{BalanceControl} find a method for balance maintaining movement. \cite{Liu2009} proposed a method for object manipulating animation.
it is adpative, but it is not numerically stable and converge slowy. initialy it is limited to low Dof system and Rigid Body.
also by some paramter curves, it is used for system with system with dynamics partial differential systemw\cite{wu2003realistic}.

%walking
%stance


\textbf{high level controller}
Faloutsos\cite{faloutsos2001composable} devleop method for combining different motion tasks.




\subsection{Biology Research}
Because of the mechanical properties and neural computational constraints,
Biological Research have different ideas about the motor control theory towards motor control\cite{latash2008neurophysiological}.
important idea in biological are
uncontrollled manifold hypothesis,
equibrium point hypothesis
and impendance control hypotheis.
basically, it means motor control is not achieved by shooting, but by the changing the system's dynamics through ajust the system's parameter\cite{feldman1986ome}.



An different idea is morphylocial computation\cite{nishikawa2007neuromechanics},which means motion is not contolled, it mainly comes from the dynamic interaction from the body and environment.
Motion is based on many basic motion patterns, which are motion primitives. neural system only modify the basic moiton patterns for application perturon.
Some research model the itearction of neural system and mechanical throught the entraintment effects\cite{Cohen1988}, 

also some research in model the neural tweaking effects as Lie Group.
Neural motion are applied by transform some motion template\cite{flash2007affine}.



\subsection{Control Theory and Robotics}


inspired by the system insipred new range of bioligical based wich inspired the natural dynamics of the enironment.
$\lambda$ model\cite{taga1991self},CPG\cite{Williamson1999}, 
Lie Group \cite{spong2005controlled} has also used for robotic research.
$\lambda$  mode and CGP can generate  adaptive motion for unpredictable.
Lie Group Method maintain the moton invaraint,but will not effect the stability.

From Geometrical Mechanical\cite{abraham1978foundations} viewport . 
They effect is closely related the topology of the mechanical system.
While Lie Group Method exloring the Symmetry of Mechanics\cite{marsden1999introduction}.





\section{Ref}
Lie group Theory \cite{olver1986applications}

mechancial model in walking \cite{chen2007passive}

mechanical model of bouncing ball
neural oscillator parameters 

stancing model is from\cite{stephens2009modeling}

for lagrange dynamics ref \cite{Goldstein2002}


fish exmaple and analysis\cite{ostrowski1995mechanics}\cite{melli2008hierarchy}


high degree motio generate from mechanical coupling \cite{Collins2009}
torso,arm and sideway dynamics ,\cite{grizzle2001asymptotically},\cite{chevallereau2009asymptotically},\cite{ames2007geometric}

