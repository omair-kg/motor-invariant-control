\section{Symmetry and Local Motion Control}
\subsection{Symmetry of Motion}
For Physically-based animation,
Motion is usually described by  the differential equation (1)
\begin{equation}
\dot{X}=F(x,u)
\end{equation}
Physically possible motion is the solution of the equation.
An important property from one solution $x$, with a group action $g$, we ca get another solution $x_a$
\begin{equation}
x_a=g_a(x).
\end{equation}
for example
, 
We have 

So the group action is


For equation (1), the group action $g_a$ satisfy the symmetry property (2).
	(3)
This provide us an idea about motion synthesis.Given an original motion m, and the corresponding group g, a new motion is generated by g(m).

For every group G, we can find an function I(x) unchanged by the group action G, 

I(x) are called local motion signature. 
For mechanical system, Lie Group and Symmetry has important physically meaning. 
I(x) corresponding to the Conservative Law, like energy or angular momentum.

\subsection{Control Symmetry}
For motion synthesis, usually the desired motion is ma and original motion m is  known, but the corresponding group action $g_a$ is not satisfied by differential equation.
For such situation, control input u  is added, which modify the original equation to allow the designed G, this is called Controlled Symmetry.

Most dynamic motion can be modelled as an Lagrange System. 
\begin{equation}
L=K(\dot(q)-V(q).
\end{equation}
And the desired action G must keep the L invariant. 

The original m is defined by the eural langrage equation
 (4)
 \begin{equation}
\frac{d}{dt}\frac{\partial L}{\partial \dot{q}}-\frac{\partial L}{\partial q} = 0
\end{equation}
The modified system is 
 (5)
 \begin{equation}
\frac{d}{dt}\frac{\partial L}{\partial G(\dot{q})}-\frac{\partial L}{G(\partial q)} = 0
\end{equation}
Which is equal the controlled dynamic system
 (6)
\begin{equation}
\frac{d}{dt}\frac{\partial L}{\partial \dot{q}}-\frac{\partial L}{\partial q} = u
\end{equation}
(5) and (6) are the equivalent equation, by comparing  equation (5) and (6), we can get u
\subsection{Examples}

\subsubsection*{Offset}
\[
G_r(x)=(q+r,\d{q})
\]
Which keep speed, but modify the pos. thus keep the K but modify V
\begin{equation}
u=\frac{\partial V(g_r(q))-V(q)}{(\partial q)}
\end{equation}
on phase space, if q is the horizontal axis, and $\dot{q}$ is the vertical axis, this has the effect of moving the phase plot right and right.

\subsubsection*{Time Scalling}
\begin{equation}
g_st(q,dot{q})=(q,st*dot{q})
\end{equation}
\begin{equation}
u=(st^2-1) \frac{\partial V(q)}{(\partial q)}
\end{equation}
on phase space, this has the effect strength the phase plot in the vertical direction

\subsubsection*{Energy Scalling}

For some system moving the the conservtime field with constant mass matrix.
The energy is preserved and different motion present different level of energy.
For such system, we have the 
For such
\[
g_e(q,\dot{q})=(e^2*q,e*\dot{q}).
\]

U can be developed by applying the pos scaling and time scaling in a combined manner.

On phase plot, this has the effect enlarge the phase portrait.
\subsection{Symmetry of CPG}
For a dynamic system L, we have global invarant controller  Ugc, and local Invariant Controller Ulc.
A question is how to combine them together.
Our method is combine the different controller following the same symmetry.
When a system L has a symmetry of $G_a$, $G_a$ is applied also tow $G_gl$
The Simmetry of Global Oscilator.
Neural oscillator are couple with mechanical oscillator in the following manner.
\begin{equation}
\dot{x}=F(x,u1)
\dot{x_c}=S(x_c,u2);
u1=ho*O(x_c)
u2=hi*I(x)
\end{equation}
Pose offset:
When the original $G_off$ is apply to the x,
If we keep $I(x)$ is the invariant of $G_off$, then the Symmetry of the how system will be kept.
Pose scalling group
When $G_scal$ is applied to the x, if we $hi=hi/scal$.
Then the symmetry is kept.
Time scalling
If the scale factor is ts, we keep modify modify the original equation by %$\tao=tao*ts$.
hou=hou*ts*ts.
Time offset
This involves modify the state on the limit circle.



