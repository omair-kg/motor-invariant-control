\subsection{Stance And Walking Transition}
\subsubsection{Dynamic Model}
the reduced model
linear bipedal model
suppose when standing, we hight vaiation is almost zero.
We only conside the horizontal movement


basicall the stance are divided into three phase

1 Single Support Phase
%ddy=t/ml+g*y/L
2 double stance Phase
%ddy=(TL+TR)/ML+g/l*wL(y-y_m)+g/l*wr(ym-yr)
3 fall 
if ym >d
than the stance will fall.

The goal of control is confine the flow withing the safe region
\subsubsection{Global And Local Motor Control}
he original system is simmilar to mass spring system.
It wil vibrate continutelly.




while in our method ,by coubling neural system the the oscillator, it form a limit circle

But because of the special characteristics of the stancing model, when move out the safe region, 
It will fall.

2 lie group control
By the speed transformation,
The speed we can control the stance motion to move within the safe region.

then the stance posture is controlled

by the energy transformation,
we can control the final vibration magnitude.
\subsection{Walking Stance Transition}
Application 
Walking and Stance Transformation



dynamic model

walk to stance transform

when transform from walk to stance, there is no heel impact.
The swing leg touch the ground, it will generate force and start to bend
x only depend on the stance leg
there is a transform f, that transform the state from q to x.
x=f(q).

stance to walk.

Stance to walk happens , the font leg is kept strait, the back leg is becoming strait and rotate.


Walk to Stance Transform

the walk state


From Stance 2 Walk

the black one is on the limit cirle, while the red one is walking start point.

