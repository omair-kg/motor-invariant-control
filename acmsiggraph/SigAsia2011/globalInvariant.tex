\section{Global Motor Invariant and Control}
\subsection{the Qualatitve Property of Motion}
When generating adaptive motion, one key question is what can be adaptive while what must be maintained.
For walking example, people walking in different manner and different way.
But still some key properties are kept unchanged.

A method for modelling the qualitative property can be identified by the topology of phase plot.
For the walking example, some researchers argue that, the key properties of walking or any other kinds of locomotion, is the periodic change of the body shape.
On the phase plot, this means the topology of the phase plot is the same to that of the circle.
The topology of the phase plot is defined as the Global invariant, which determines the qualitative properties of motion.

When keep the topology against the change in system dynamic, this is the structuable stability control, which in nature solves the motion retargeting problem.
When the topology is kept against the perturbation in state, this is the stability control,
Which in nature solves the stalibty control in motion synthesis.

Different kinds of perturbation will generate different shape of phase plot. This is allowed in our framework and even seen as an advantage, because the shape changed reflect the adaptation in motion. 


By the topology,
basically motion can be separated into two group.
1 periodic motion
Some motion show pepetive pattern, from geometrical viewport, on the phaseplane, the attractor of curve form the shape of a circle.
2 discrete motion
Some motion is terminated, from geometrical viewport, on phase phaseplane,  a ttractor is a point.

In this research, we only consider periodic motion. This for two reason, the first one is that some biological research believe that life system itself is peroditic in nature. From application viewport, fix point can be approximate with a limit circle with a small amplitude.

\subsection{Entrainment}
For periodic motion, global controller don’t care about the shape, of oscillation, it only cares about the topology of the phase plot.
The method we prosposed is based on the entrainment.
The basic idea is couple two oscillator together. The dynamic system form one oscillator, while the control system form the other oscillator.
The control system system is topological structural stabile, and oscilate with the mechanical one in an similar manner,
When the mechanical oscillator is disrupted, whe control oscillator will drive the mechanical oscillator return to oscillation model.
Whe control oscillatior we prosed is based on the matusta oscillator.
With the equation of following form.


Stability of Control Oscillation.
Automotous Oscillation
From initial position, matsushiata oscillator all converge to the same limit circle.
Entrainment Oscillation
Matsuta oscillation can coupble with a various oscillator.
\begin{eqnarray}
\tau_{1} \dot{x_{1}}&=&c-x_{1}-\beta v_{1}-\gamma [x_{2}]^{+}-\sum_{j}h_{j}[g_{j}]^{+}\\
\tau_{2} \dot{v_{1}}&=&[x_{1}]^{+}-v_{1}\\
\tau_{1} \dot{x_{2}}&=&c-x_{2}-\beta v_{2}-\gamma [x_{1}]^{-}-\sum_{j}h_{j}[g_{j}]^{-}\\
\tau_{2} \dot{v_{2}}&=&[x_{2}]^{+}-v_{2}\\
y_{i}&=&\mbox{max}(x_{i},0)\\
y_{out}&=&[x_{1}]^{+}-[x_{2}]^{+}=y_{1}-y{2}
\label{eq:matsuta}
\end{eqnarray}